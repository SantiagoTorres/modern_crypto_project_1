\documentclass[pdftex,12pt,letter]{article}
\usepackage[pdftex]{graphicx}

\usepackage{pdfpages} 
\usepackage{fixltx2e}

\addtolength{\textwidth}{3.5cm}
\addtolength{\hoffset}{-2cm}
\addtolength{\textheight}{4.5cm}
\addtolength{\voffset}{-3cm}

\bibliographystyle{plain}
\usepackage{natbib}
\newcommand{\HRule}{\rule{\linewidth}{0.5mm}}

\renewcommand{\thesection}{\Alph{section}}

%\usepackage{blindtext}
\usepackage{listings}

\usepackage{amsmath}

\usepackage{subfig}
\begin{document}
\begin{titlepage}

\begin{center}


% Upper part of the page
\includegraphics[width=.5\textwidth]{./logo}\\[1cm]    

\textsc{\LARGE CS-GY 6903: Modern Cryptography}\\[1.5cm]

\textsc{\Large Professor: Giovanni Di Crescenzo}\\[0.5cm]

% Title
\HRule \\[0.4cm]
{ \large \bfseries Project 1:\\
{\small Breaking Polyalphabetic ciphers}}\\[0.4cm]

\HRule \\[1.5cm]
% first column
\begin{minipage}{0.4\textwidth}
\begin{flushleft}
\large Santiago Torres-Arias\\
{\small {\bf NYU ID: N14553751 } }
\end{flushleft}
\end{minipage}
%second column
\begin{minipage}{0.40\textwidth}
\begin{flushright}
\large Lucas Mladek\\
{\small {\bf NYU ID: N1xxxxxx } } 
\end{flushright}
\end{minipage} 

\vfill

% Bottom of the page
{\large \today}

\end{center}
\end{titlepage}

\newpage
\tableofcontents
\newpage

\section{Introduction}
Polyalphabetic ciphers have been around in humanity for hundreds of years now.
And, although their use is no longer recommended for real cryptography, its use
is still widespread among cryptography amateurs and cryptography
challenges.

It is known that polyalphabetic ciphers are still subject to frequency
analysis, and known plaintext attacks. The focus of this writeup is to
introduce techniques that leverage this knowledge to break these ciphers in the
least possible time.

\section{Understanding polyalphabetic ciphers}
% Here, we talk about what's a polyalphabetic cipher
% and how the guy must be high about thinking of the j(x) function


\subsection{The anatomy of the $j(i)$ function}

We can try to decompose the $j(i)$ function by using calculus concepts. The
following equation provides a simplified version of the possible variants that
$j(i)$ can have:

\begin{equation}
    j(i) = \left( \frac{xi^y}{z} + m \right) \mod{t}
\end{equation}

In this equation, we adopt the following nomenclature:
\begin{itemize}
    \item i is the index
    \item x is a define a ``decimating coefficient'' for i.
    \item y defines a ``decimating exponential'' factor for the index. (we assume y = 1)
    \item The constant term $z$ defines a ``stretching factor''
    \item m is a ``start offset'', and it can be described as shifting the key by m bins.
    \item $\mod{t}$ guarantees that the values selected are inside the generated key and t is the length of the key.
    \item we assume $\frac{x}{z}$ is a multiple of t, this makes it periodic on t.
\end{itemize}

The relationship between the upper and lower factors define how this function behaves. 
In general, we can consider to be three variants: periodic, decimating and stretching. 

\subsection{Key stretching and key-decimating}

When the factor above ($xi^y$) is polinomially bigger than the factor below
($z$), then we can assume that we have a key-decimating function. Key
decimating functions make the key essentially "smaller" in a sense that some
values are skipped and hence lost (depending on the periodicity).

On the other hand, when values from the upper monomial are smaller than the
lower one, we have what's called a key-stretching function. When this happens,
we see that -- due to integer math -- some values are repeated contiguously.
For example, imagine that we have the following $j(i)$:

\begin{equation}
    j(i) = \frac{i}{4}\mod{30}
\end{equation}

In this case, the resulting function can be considered stretching because
values for i below 4 will all point to 0, which is the first element of the
key.

For key-stretching functions, we will be unable to decrypt the ciphertext. 

For key-decimating functions, we will be able to decrypt the ciphertext 
 if $\frac{x}{z}$ is a factor of t and therefore periodic on t. 

The nature of this equation has a direct effect on what we will call the key's
periodicity.

\subsection{Periodicity}

Periodicity of the key (and $j(i)$) can be understood as ``the number of values for i
before the key-selecting sequence repeats''. Understanding this can allow us for 
an easy definition of a break function. 

We will define 'regular periodic functions' as those functions in which $\frac{x}{z}$
is a multiple of t. When this happens, the periodicity of $j(i)$ also falls
in a multiple of t. For our breaking scheme, we considered the function $j(i)$ to  
be a regular periodic function since it was the easiest to analyze. 

However, we could break Non-regular periodic functions by finding the same subset of the key in different places 
along the ciphertext. This is a regular step in breaking polyalphabetic cihpers if you are 
not given the key length. Since we were given the key length we did not take this approach.   

\subsection{Why periodicity matters and $j(i)$ not as much}

If we consider a function to be regular periodic, then we can assume that the 
set of keys ${a...z}^t$ under a specific $j(i)$ is only a different permutation
of another set of keys under a different $j(i)$. Knowing this, we can
assume that, when a function is regular periodic, obtaining a proposed key for
$j(i) = i \mod(t)$ is a safe bet. We will use this assumption in our breaking 
mechanisms, as it simplifies our analysis greatly. Refer to figure~\ref{key-sets} 
for a graphic description of this.

\begin{figure}[ht!]
    \centering
    \includegraphics[width=.8\textwidth]{key-mapping}
    \caption{Different sets are mapped using different functions $j(i)$ but,
    if $j(i)$ and $j'(i)$ share periods, both sets will share a large amount of keys.}
    \label{key-sets}
\end{figure}

\subsection{Frequency analysis on polyalphabetic ciphers}

In most cases frequency analysis breaks polyalphabetic cihpers. However,
for this project the cihper text we are able to sample is too small to 
perform frequency analysis on. 



Kasiski Examination is another technique that would work and is similar to 
a frequency analysis but also requires us to be able to sample a greater amount 
of ciphertext.


\section{Breaking dictionary 1}

In order to design the most efficient mechanism to break the defined cipher, we 
first need to understand the nature of the existing dictionaries. In this case
we know that the existing plaintexts falls in a range of 150 different plaintexts
so we decided to analyse them as words.

This is known as a 'known plaintext attack'. The dictionaries given were a small 
enough set to be analyzed using the methods outlined in this section. 

\subsection{Minimum prefix: plaintexts as words}

Since the set of possible messages from Dictionary one is the same as the number of 
entries a dictionary 1, cryptanalysis is really easy. We could consider that the 
messages from dictionary one are a single, long word encrypted and sent through the wire.

To understand this, we build a python script called "minimum\_prefix.py", that analyses the 
words in a dictionary to identify the minimum amount of characters needed to differentiate
plaintext from each other. The result of this script output a value of 9, and we used it
as part of our dictionary-header definition; this script can be found in the sources.


We can obtain a possible key by substracting a piece of plaintext from the ciphertext:

\begin{equation}
    p_{[i]} + k_{[x]} = c_{[i]} \rightarrow k_{[x]} = c_{[i]} - p_{[i]}
\end{equation}

Now, how do we know our key is correct?

\subsection{Defining our Oracle and Attack}

Breaking dictionary 1 is easy: we only need to apply the prefix plaintext to
the provided ciphertext and compare the resulting  key to another portion of
the plaintext. The only catch is that we need to identify a region that uses
the same range of the key. Since we assumed $j(i)$ to be regular periodic, then
we can assume that it is a place where period starts and verify if the result
matches correct plaintext of the same dictionary entry in that location. 

We used a few 'speed-ups' in our code, for example declaring the entire
dictionary as a constant in C, alphabetizing the the dictionary, and 
using a 'trampolining' function to jump the the first letter of the
alphabetized potential plaintext dictionary.  

The code we wrote for this takes less than a second to verify and dismiss all
possible candidates for the plaintext. 

\begin{figure}[ht!]
    \centering
    \includegraphics[width=.8\textwidth]{breaking-dict1}
    \caption{The breaking procedure from ciphertexts produced from Dictionary 1}
    \label{breaking-dict1}
\end{figure}

\section{Breaking dictionary 2}

For dictionary 2 we can't use the same exact method as dictionary 1.
So we researched different possible methods and case studies that were 
relevant to our attack. Before we explain our attack we will give a 
brief explanation of one such case study that help formulate our attack
strategy.  

\subsection{Case study: Rosignol Cipher - The man with the Iron mask}
https://en.wikipedia.org/wiki/Great\_Cipher

The Rosignol's were a family of French cryptographers in the 17th century.
The most famous of the family was Antoine Rossignol and his son Bonaventure
who in ~1626 developed a cryptographic code so strong it baffled cryptanalysts for 
centuries (wasn't decoded until 1893). This cipher used numbers to represent syllables 
rather than individual characters (the common approach at that time). The technical 
nature of the cipher includes a small 587 set of numbers that represent syllables 
and by itself is still subject to frequency analysis attacks. 

In addition, this Cipher Schema seemed to pose a decryption/solution to solve the 
mystery of the man in the iron mask (infamous french prisoner at the time). 

This inspired the approach we took to cracking dictionary 2. We focused on 
tuples of characters (like syllables but not quite) and their possible combinations
in order to determine partials of words, or concatentations of 2 different words. 


\subsection{Breaking polyalphabetic ciphers using N-character tuples (trillables)}

\subsubsection{The size of the set}
The size of the set is important to note for breaking dictionary 2 using tuples. 
Set groups as we make tuples bigger grow, but the probability of them appearing
in the plaintext decreases. We found 3 tuples to be of an optimal size for our
purposes and thus have dubbed them triads or trillables. 
\subsubsection{Probabilistic results}
Every time we decrypt a 3-character tuple(trillable) there is a 
probability(of $\approx$ .1907) that it belongs to the subset of all possible trillables. 

In addition as the number of trials(in our case each trial is a decryption attempt) 
increases, the probability that all of the trillables fall into the set of possible trillables 
decreases.  

Theoretically there are possible collisions that don't make this a fully encompassing approach. 
However, this method is much faster so we are trading quality for performance in order to make 
our overall decryption as fast as possible.

\subsection{Defining our Oracle and Attack} 
We perform the same attack as in dictionary 1, however our oracle has changed. 
We now use the 3-character tuples (trillables) approach as defined in section D.2.2 
 
\begin{figure}[ht!]
    \centering
    \includegraphics[width=.8\textwidth]{breaking-dict2}
    \caption{The breaking procedure from ciphertexts produced from Dictionary 2. Note
    how it's possible that the second decrypted subsection is part of a word, or even two or more
words}
    \label{breaking-dict1}
\end{figure}


\section{Results}
We made a script that randomly selects a $j(i)$ with our constraints listed above and 
encrypts a plaintext from either dictionary 1 or 2. We used the results of this script 
to test our decryption program.  

We utilized benchmarking functions and unit testing in our code for efficiency 
and troubleshooting respectively. 

We also implemented a stopwatch timer in order to stop decryption attempts if 
our program began to run our of time. However our decryption schema/program runs 
so quickly we never needed to utilize it. 

We found that our program can decrypt the ciphertext generated by our script in 
an extremely short amount of time relative to the 2 minutes upper limit for the 
assignment. 

Lastly, we found no collisions that break our attack schema although admittedly 
they could indeed exist and would not permit for our attack against dictionary 2. 
 
\section{Conclusion}
We researched and found different and creative ways for breaking polyalphabetic ciphers.
Some of which were implementable but not within the given time period for the project.
Others were infeasible due to the dictionary being relatively short. Our attacks 
greatly depended on the fact the the key length was given and the plaintext was
given beforehand (known plaintext attack).

Both partners Santiago and Luke were involved in the following tasks: 
Outlining and Testing  Attack Strategies, Software Engineering and Coding, Analysis of Results, 
Write-Up and Final Analysis.

\end{document}
